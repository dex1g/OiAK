\documentclass{article}
\usepackage[utf8]{inputenc}
\usepackage[T1]{fontenc}
\usepackage{natbib}
\usepackage{graphicx}
\usepackage[a4paper,top=2cm, total={6in, 8in}]{geometry}


\title{Biblioteka arytmetyki liczb stałoprzecinkowych dowolnej precyzji z wykorzystaniem wewnętrznej reprezentacji U2.}

\author{Leszek Błażewski \\ \\Karol Noga}

\date{Semestr letni 2018/2019}

\begin{document}
\maketitle
\clearpage
\section{Główne założenia projektu}
Głównym założeniem projektowym jest implementacji biblioteki liczb w języku C o dowolnej precyzji, która wykorzystuje system U2 jako bazowy. Aby możliwe było osiągnięcie nieskończonej dokładności postanowiliśmy, że dane będą przechowywane w postaci tekstowej. W języku C reprezentowane są one przy pomocy dynamicznie alokowanej tablicy charów.

\section{Konwersje}
\subsection{Opis}
Zgodnie z założeniami projektu biblioteka musi operować na danych, których wartości wyrażone są w systemie U2. Liczby dodatnie nie posiadają znaku natomiast liczby ujemne przed konwersją reprezentowane są z znakiem minus.
W zależności od wartości ulegającej konwersji liczba reprezentowana jest w systemie U2 z odpowiadającym jej znakiem.

\vspace{5mm} %5mm vertical space

Poniżej zamieszczono listę konwersji do systemu U2, które oferowane są przez bibliotekę:
\begin{itemize}
    \item Konwersja z systemu 16
    \item Konwersja z systemu 8
    \item Konwersja z systemu binarnego
    \item Konwersja z systemu 10
\end{itemize}

\subsection{Algorytm konwersji}
W celu wyznaczenia wartości liczby zapisanej w systemie binarnym z znakiem, przechowywanej jako tekst posłużono się prostym algorytmem polegającym na skanowaniu liczby od tyłu w celu znalezienia pierwszej "1" występującej w liczbie oraz negacji wszystkich następnych bitów.

\vspace{5mm} %5mm vertical space

W skrócie algorytm można zapisać następująco:
\begin{enumerate}
    \item Skanuj liczbę od tyłu spisując wszystkie "0" aż do momentu wystąpienia pierwszej "1".
    \item Znalezioną liczbę również przepisz do wyniku.
    \item Wszystkie kolejne bity liczby konwertowanej neguj.
    \item Jeśli liczba nie ma znaku - na początku dopisz "0" z przodu liczby w przeciwnym wypadku dopisz "1".
\end{enumerate}

Liczby zapisane w innych systemach najpierw konwertowane są na reprezentację binarną z znakiem, a następnie według powyższego algorytmu konwertowane są na sytem U2.

\section{Podstawowe operacje arytmetyczne}
Biblioteka oferuje następujące operacje arytmetyczne na danych reprezentowanych w systemie U2. Operacje te zostaną zaimplementowane przy użyciu instrukcji asemblerowych.

\vspace{5mm} %5mm vertical space

Lista podstawowych operacji artmetycznych:
\begin{itemize}
    \item Dodawanie
    \item Odejmowanie
    \item Mnożenie
    \item Dzielenie
    \item Pierwiastkowanie
    \item Potęgowanie
\end{itemize}

\section{Problemy jakie napotkaliśmy dotychczas}
\begin{enumerate}
    \item Czy reprezentacja w tablicach charów jest odpowiednia. Jeśli nie czy są jakieś lepsze alternatywy ?
    \item Jak konwertować liczby z systemu dziesiętnego (trzymane w tablicy charów) na binarny, (część całkowita + ułamkowa), jak usprawnić ten algorytm ?
    \item Czy założenie że precyzja liczby jest określana na podstawie wprowadzonej danej jest odpowiednia ?
    \item Czy dokładność wyniku działania powinna być zależna od składnika działania który miał większą precyzję ?
    \item Co zrobić w przypadku algorytmu pierwiastkowania gdy liczba ma okres który jest bardzo długi i trudno wykrywalny  lub nie jest okresowa ?
    \item Czy wykonują kod asemblerowy lepiej przekształcić tablice charow z ascii na faktyczne liczby czy zostawić i wykonywać operacje na kodach.
\end{enumerate}
\end{document}
