\documentclass[a4paper]{article}
\usepackage[utf8]{inputenc}
\usepackage[T1]{fontenc}
\usepackage{natbib}
\usepackage{graphicx}
\usepackage[a4paper,top=2cm, total={6in, 8in}]{geometry}


\title{Biblioteka arytmetyki liczb stałoprzecinkowych dowolnej precyzji z wykorzystaniem wewnętrznej reprezentacji U2.}

\author{Karol Noga, 241259 \\ \\Leszek Błażewski, 241264}

\date{Semestr letni 2018/2019}

\begin{document}
\maketitle
\vspace{1mm}
{\centering \Large{4.04.2019}\par}
\vspace{1mm}
{\centering \Large{Środa TP 13:15}\par}
\clearpage
\section{Sposób reprezentacji liczby}

Podczas implementacji operacji na liczbach, zauważyliśmy, że w celu przyśpieszenia wykonywanych operacji, liczby przetrzymywać możemy w postaci tablicy integerów, dzięki czemu operacja dodawania wykorzystać może maximum możliwości procesora wypełniając cały rejestr daną liczbą.


\section{Konwersja z użyciem instrukcji asemblerowych}
 
\section{Wyznaczanie wartości liczby w systemie U2}

\subsection{Algorytm konwersji}

\section{Testy jednostkowe}


\section{Sposób przechowywania liczb}


\section{Problemy jakie napotkaliśmy w realizowanym etapie}
\begin{enumerate}
    \item Czy zapamiętywanie pozycji przecinka w liczbie w kodzie ascii i pomijanie go w trakcie konwersji na bajty, a przy instrukcjach assemblerowych odpowiednie skalowanie liczby względem zapamiętanego miejsca przecinka jest dobrym rozwiązaniem ?
    \item Jak poradzić sobie z konwersją liczb dziesiętnych zapisanych w postaci znaków ascii ? 
\end{enumerate}
\end{document}
